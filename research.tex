\documentclass[]{article}
\usepackage{lmodern}
\usepackage{amssymb,amsmath}
\usepackage{ifxetex,ifluatex}
\usepackage{fixltx2e} % provides \textsubscript
\ifnum 0\ifxetex 1\fi\ifluatex 1\fi=0 % if pdftex
  \usepackage[T1]{fontenc}
  \usepackage[utf8]{inputenc}
\else % if luatex or xelatex
  \ifxetex
    \usepackage{mathspec}
  \else
    \usepackage{fontspec}
  \fi
  \defaultfontfeatures{Ligatures=TeX,Scale=MatchLowercase}
\fi
% use upquote if available, for straight quotes in verbatim environments
\IfFileExists{upquote.sty}{\usepackage{upquote}}{}
% use microtype if available
\IfFileExists{microtype.sty}{%
\usepackage{microtype}
\UseMicrotypeSet[protrusion]{basicmath} % disable protrusion for tt fonts
}{}
\usepackage[margin=1in]{geometry}
\usepackage{hyperref}
\hypersetup{unicode=true,
            pdfborder={0 0 0},
            breaklinks=true}
\urlstyle{same}  % don't use monospace font for urls
\usepackage{graphicx,grffile}
\makeatletter
\def\maxwidth{\ifdim\Gin@nat@width>\linewidth\linewidth\else\Gin@nat@width\fi}
\def\maxheight{\ifdim\Gin@nat@height>\textheight\textheight\else\Gin@nat@height\fi}
\makeatother
% Scale images if necessary, so that they will not overflow the page
% margins by default, and it is still possible to overwrite the defaults
% using explicit options in \includegraphics[width, height, ...]{}
\setkeys{Gin}{width=\maxwidth,height=\maxheight,keepaspectratio}
\IfFileExists{parskip.sty}{%
\usepackage{parskip}
}{% else
\setlength{\parindent}{0pt}
\setlength{\parskip}{6pt plus 2pt minus 1pt}
}
\setlength{\emergencystretch}{3em}  % prevent overfull lines
\providecommand{\tightlist}{%
  \setlength{\itemsep}{0pt}\setlength{\parskip}{0pt}}
\setcounter{secnumdepth}{0}
% Redefines (sub)paragraphs to behave more like sections
\ifx\paragraph\undefined\else
\let\oldparagraph\paragraph
\renewcommand{\paragraph}[1]{\oldparagraph{#1}\mbox{}}
\fi
\ifx\subparagraph\undefined\else
\let\oldsubparagraph\subparagraph
\renewcommand{\subparagraph}[1]{\oldsubparagraph{#1}\mbox{}}
\fi

%%% Use protect on footnotes to avoid problems with footnotes in titles
\let\rmarkdownfootnote\footnote%
\def\footnote{\protect\rmarkdownfootnote}

%%% Change title format to be more compact
\usepackage{titling}

% Create subtitle command for use in maketitle
\providecommand{\subtitle}[1]{
  \posttitle{
    \begin{center}\large#1\end{center}
    }
}

\setlength{\droptitle}{-2em}

  \title{}
    \pretitle{\vspace{\droptitle}}
  \posttitle{}
    \author{}
    \preauthor{}\postauthor{}
    \date{}
    \predate{}\postdate{}
  

\begin{document}

{
\setcounter{tocdepth}{3}
\tableofcontents
}
\hypertarget{rshiny-app-development}{%
\section{RShiny App development}\label{rshiny-app-development}}

\hypertarget{parameterization-of-red-snapper-individual-based-simulation-model}{%
\subsubsection{Parameterization of Red Snapper Individual-Based
Simulation
Model}\label{parameterization-of-red-snapper-individual-based-simulation-model}}

To understand the feasiblity of performing large-scale tag-recapture in
the Gulf of Mexico. To understand the challenges we distributed this
survey to scientists and those knowledgable in Red Snapper ecology:

Thanks very much for taking part in this effort to understand the
ecology of Red Snapper. We are interested in how these dynamics impact
our ability to determine the abundance of the stock in the nothern Gulf
of Mexico. Below are a series of questions that when answered will allow
us to better model population and individual dynamics of Red Snapper.

We are modeling the movement dynamics of snapper with a spatial network
and seek to understand how frequently and under what circumstances fish
will move to another habitat (displace), habitat characteristics, and
tagging and recapture dynamics

In each of the fields below will be a series of sliders and buttons.
Please interact with these to give your best estimates to the queries,
based on your knowledge of Red Snapper. We understand that some of the
queries may be difficult and indeed not known by the scientific
community. We will be simulating the dynamics over a range of scenarios
to capture this ambiguity - your thoughtful efforts will help us to
understand the extent of current scientific knowledge and also
understand areas where more research is needed.

Associated with many of the queries will be a field where you report the
amount of confidence you have in your estimate - these will be used as
criteria to partially weight your response to those of your peers. At
the end of the page will be a download button so that your answers can
be submitted to our team.

\href{https://robe.shinyapps.io/SurveyRedSnapper02/}{Parameterization of
Red Snapper Individual-Based Simulation Model}

\hypertarget{trophic-webs-of-the-gulf-of-mexico}{%
\subsubsection{Trophic Webs of the Gulf of
Mexico}\label{trophic-webs-of-the-gulf-of-mexico}}

This application was developed for the work of Oshima and Leaf 2018. We
collected diet studies from the northern Gulf of Mexico and conducted
network and simulation analyses to evaluate the structure and robustness
of the trophic dynamics. This app allows the user to select either a
predator or prey, the diet metric used to report the interactions and
the taxonomic level the nodes are shown at. Note that prey were
identified to varrying taxonomic levels depending on the studies. If you
do not see a network displayed or get an error, try selecting a
different diet metric. The diet metrics reported for a selected predator
or prey can be seen in the data table.

\href{https://megumi-oshima.shinyapps.io/diet/}{Diet Network in the Gulf
of Mexico}

\hypertarget{student-judging-algorithm-and-analysis}{%
\subsubsection{Student Judging Algorithm and
Analysis}\label{student-judging-algorithm-and-analysis}}

Shiny application for aggregating judges scores. csv files are generated
and then synthesized in .RMD reports for students. Student Summary
Report for Presentation is then distributed and includes 1. Student
Score, 2. Evaluation Criteria and Weights, and 3. Your Performance in
each of the Evaluation Criteria.

\href{https://robe.shinyapps.io/judge/}{Student Judging}

\hypertarget{current-and-past-research-projects}{%
\section{Current and Past Research
Projects}\label{current-and-past-research-projects}}

\hypertarget{escapement-rate-of-red-drum}{%
\subsubsection{Escapement Rate of Red
Drum}\label{escapement-rate-of-red-drum}}

This objective of this research effort is to estimate mortality
components and escapement rate of Red Drum stock in Mississippi. This
project consists of determinations of instantaneous natural mortality
rate, evaluation of overall and year-specific instantaneous total
mortality rates, and calculation of the overall and year-specific
instantaneous fishing mortality rates of Red Drum in Mississippi.
Mortality estimates are then used to calculate an escapement rate.
Alternative natural mortality estimations and fishery recruitment
criteria are also evaluated to elucidate the impact of model
misspecification and how these decisions affect the escapement rate.

\hypertarget{age-and-growth-of-red-drum}{%
\subsubsection{Age and Growth of Red
Drum}\label{age-and-growth-of-red-drum}}

The objective of this research effort is to evaluate the length-at-age
and weight-at-length relationships of Red Drum. Sex-specific
length-at-age and sex-specific weight-at-length relationships are
quantified using multiple candidate models to determine the best
supported model. Sex-specific relationships were examined for
significant differences in male and female Red Drum.

\hypertarget{reproductive-biology-of-red-drum}{%
\subsubsection{Reproductive Biology of Red
Drum}\label{reproductive-biology-of-red-drum}}

The objective of this research effort is to use the most accurate and
current methods available to determine the reproductive characteristics
of Red Drum, including (1) the spawning season, (2) the sex-specific age
and length at maturity, and (3) the spawning interval.

\hypertarget{detecting-residency-and-emigration-of-red-drum}{%
\subsubsection{Detecting Residency and Emigration of Red
Drum}\label{detecting-residency-and-emigration-of-red-drum}}

The objective of this research effort is to use tissue specific carbon
(δ13C) and nitrogen (δ13N) isotope composition to determine ontogenetic
movement of Red Drum through coastal and marine systems. Differences in
stable isotope composition across year age classes with respect to
reproductive phase and location of catch will help better determine
ontogenetic movement of Red Drum between estuarine and marine systems.

\hypertarget{size-and-age-distribution-of-gulf-menhaden}{%
\subsubsection{Size and Age Distribution of Gulf
Menhaden}\label{size-and-age-distribution-of-gulf-menhaden}}

The objective of this research effort is to use the Gulf of Mexico
states' monthly fishery independent data collection efforts to
understand temporal and spatial dynamics of the Gulf Menhaden fishery.
Length and weight data are examined for individuals collected by state
agencies in Texas, Louisiana, Mississippi, and Alabama. Age of Gulf
Menhaden is determined blindly using whole, polished otoliths and scale
samples. This work will also be used to form recommendations for
methodology improvements to enhance ageing precision.

\hypertarget{network-analysis-of-trophic-dynamics-in-the-northern-gulf-of-mexico}{%
\subsubsection{Network Analysis of Trophic Dynamics in the Northern Gulf
of
Mexico}\label{network-analysis-of-trophic-dynamics-in-the-northern-gulf-of-mexico}}

The objectives of this research effort are to create a trophic network
for the northern Gulf of Mexico using historical literature. Network
analysis techniques are used to test the resilience of the trophic
dynamics to perturbations. We also will use these techniques to examine
how predator feeding effort may be reallocated among other prey when
prey items are reduced. This work will provide and understanding of the
complex trophic dynamics in the Gulf and highlight the cascading effects
of perturbations in the system.

\hypertarget{heirarchical-bayesian-surplus-production-model-for-blue-crab}{%
\subsubsection{Heirarchical Bayesian Surplus Production Model for Blue
Crab}\label{heirarchical-bayesian-surplus-production-model-for-blue-crab}}

The objectives of this research effort are to develop a heirarchical
Bayesian surplus production model in the northern Gulf of Mexico to
assist with stock assessments of Blue Crab. Currently, inshore Blue Crab
are assessed and managed at a state level, however, there is evidence
that there is only one stock. This model model uses indices of abundance
and catch data from different basins or states across the Gulf to
estimate a total abundance of Blue Crab in the northern Gulf of Mexico.
The heirarchical Bayesian framework is flexible, allows easy
incorporation of multiple data sets, and allows for the direct
estimation of missing data or errors. The sensitivity of the analysis is
tested to the exclusion of each index of abundance.

\hypertarget{age-and-growth-characteristics-of-atlantic-chub-mackerel}{%
\subsubsection{Age and Growth Characteristics of Atlantic Chub
Mackerel}\label{age-and-growth-characteristics-of-atlantic-chub-mackerel}}

The objective of this research effort is to describe age and growth
characteristics of Atlantic Chub Mackerel from the coastal Mid-Atlantic
and New England region of the United States. Age estimates from both
whole and sectioned otoliths are evaluated to determine which method
results in highest precision. Otolith-derived age estimates are then
used to evaluate the length-at-age relationship using a suite of
non-linear growth models. The weight-at-length relationship is modeled
using a power function. Median growth parameter estimates of Atlantic
Chub Mackerel from the northwest Atlantic are compared with mean
parameter estimates reported from other regions in the Atlantic and
Mediterranean.

\hypertarget{reproductive-dynamics-of-atlantic-chub-mackerel}{%
\subsubsection{Reproductive Dynamics of Atlantic Chub
Mackerel}\label{reproductive-dynamics-of-atlantic-chub-mackerel}}

The objective of this research effort is to describe elements of the
reproductive biology of Atlantic Chub Mackerel in the northwest
Atlantic. Histological techniques are used to determine sex and length-
and age-specific sexual maturity. The approximate spawning season is
identified using histological indicators and evidence of spawning from
analysis of historical data on larval fish collections and commercial
catch. These data will be used to inform the effective management of
Atlantic Chub Mackerel stock.

\hypertarget{evaluating-enhancement-strategies-for-spotted-seatrout-cynoscion-nebulosus}{%
\subsubsection{Evaluating Enhancement Strategies for Spotted Seatrout
(Cynoscion
nebulosus)}\label{evaluating-enhancement-strategies-for-spotted-seatrout-cynoscion-nebulosus}}

The objective of this research effort is to use a simulation-based
analysis to evaluate the efficacy of stock enhancement for Spotted
Seatrout in the north-central Gulf of Mexico. An operating model is
developed to represent the biological characteristics of the stocks and
attributes of the associated recreational fisheries. The results from
this work will provide insight into the magnitude of effort needed to
enhance Spotted Seatrout stocks and the associated costs.

\hypertarget{management-strategy-evaluation-for-vermilion-snapper}{%
\subsubsection{Management Strategy Evaluation for Vermilion
Snapper}\label{management-strategy-evaluation-for-vermilion-snapper}}

The objectives of this research effort are to develop a management
strategy evaluation for Vermilion Snapper in the northern Gulf of
Mexico. A suite of operating models will be used to represent the true
underlying population dynamics of the stock and the major uncertainties
associated with that stock, such as stock-recruit relationship. The
operating models will be used to test a set of proposed management
strategies and performance statistics will be used to evaluate how well
or poorly the stock responded to each strategy. This methodology can be
used to evaluate trade-offs of different management regulations and how
variable the outcome may be prior to implementing any regulations. It
will provide a tool for informing future management and identifying
areas where more research effort and improved data would enhance
assessment efforts.


\end{document}
